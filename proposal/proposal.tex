\documentclass{scrartcl}

\usepackage{siunitx}
\usepackage{graphicx}
\usepackage{caption}
\usepackage{subcaption}
\usepackage{glossaries}
\usepackage[english]{babel}
\usepackage{booktabs}
\usepackage[linktoc=all,hidelinks]{hyperref}
\usepackage{cleveref}
\usepackage{authblk}

% \makeatletter
% \renewcommand\AB@affilsepx{,~ \protect\Affilfont}
% \makeatother

\newcommand{\nump}[2]{\num[round-mode=places,round-precision=#2]{#1}}
\DeclareGraphicsExtensions{.pdf,.eps}
\bibliographystyle{unsrt}

\title{Proposal of Project}
\subtitle{A study of machine learning algorithms for reconstruction of missing mass in particle physics experiments.}
% \subject{Statistical Machine Learning}

\author[1]{Max Isacsson}
\author[2]{Mikael M\aa rtensson}
\author[3]{Camila Rangel Smith}
\author[4]{Henrik \"{O}hman}
\affil[1]{\small\url{max.isacsson@physics.uu.se}}
\affil[2]{\url{mikael.martensson@physics.uu.se}}
\affil[3]{\url{camila.rangel@physics.uu.se}}
\affil[4]{\url{ohman@cern.ch}}

\newacronym{ANN}{ANN}{Artificial Neural Network}

\begin{document}
\maketitle

\section{Problem statement}


\section{Background}
\subsection{Particle physics experiments}

\subsection{Charged Higgs}
% Production and decay
\subsection{Missing mass}


\section{The Dataset}
\subsection{Structure}

\subsection{Production}


\section{Solution strategy}
We will perform a 
% Regression
% Artificial neural network, bayesian regression, support vector machine, gaussian process.


%\newpage
% \bibliography{ref}

\end{document}
